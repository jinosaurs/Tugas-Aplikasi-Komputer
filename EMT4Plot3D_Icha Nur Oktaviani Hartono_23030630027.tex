\documentclass{article}

\usepackage{eumat}

\begin{document}
\begin{eulernotebook}
\eulerheading{Menggambar Plot 3D dengan EMT}
\begin{eulercomment}
Ini adalah pengenalan plot 3D di Euler. Kita memerlukan plot 3D untuk
memvisualisasikan fungsi dua variabel.

Euler menggambar fungsi tersebut menggunakan algoritma pengurutan
untuk menyembunyikan bagian-bagian di latar belakang. Secara umum,
Euler menggunakan proyeksi pusat. Standarnya adalah dari kuadran x-y
positif ke arah titik asal x=y=z=0, tetapi sudut=0° terlihat dari arah
sumbu y. Sudut pandang dan ketinggian dapat diubah.

Euler dapat memetakan:\\
- permukaan dengan bayangan dan garis level atau rentang level,\\
- awan titik-titik,\\
-kurva parametrik,\\
- permukaan implisit.

Plot 3D dari sebuah fungsi menggunakan plot3d. Cara termudah adalah
memplot ekspresi dalam x dan y. Parameter r mengatur rentang plot di
sekitar (0,0).\\
Translated with DeepL.com (free version)
\end{eulercomment}
\begin{eulerprompt}
>aspect(1.5); plot3d("x^2+sin(y)",-5,5,0,6*pi):
\end{eulerprompt}
\eulerimg{17}{images/EMT4Plot3D_Icha Nur Oktaviani Hartono_23030630027-001.png}
\begin{eulerprompt}
>plot3d("x^2+x*sin(y)",-5,5,0,6*pi):
\end{eulerprompt}
\eulerimg{17}{images/EMT4Plot3D_Icha Nur Oktaviani Hartono_23030630027-002.png}
\begin{eulercomment}
Silakan lakukan modifikasi agar gambar "talang bergelombang" tersebut
tidak lurus melainkan melengkung/melingkar, baik melingkar secara
mendatar maupun melingkar turun/naik (seperti papan peluncur pada
kolam renang. Temukan rumusnya.
\end{eulercomment}
\eulerheading{Fungsi Dua Variabel}
\begin{eulercomment}
Untuk grafik suatu fungsi, gunakan:\\
- Ekspresi sederhana dalam x dan y,\\
- Nama fungsi dari dua variabel\\
- Atau matriks data

Standarnya adalah kisi-kisi kawat yang terisi dengan warna yang
berbeda di kedua siis. Perhatikan bahwa jumlah default interval grid
adalah 10, namun plot menggunakan jumlah default 40x40 persegi panjang
untuk membangun permukaan. Hal ini dapat diubah.

- n=40, n=[40,40]:jumlah garis kisi di setiap arah\\
- grid=10, grid=[10,10]:jumlah garis grid di setiap arah.

Disini kita menggunakan default n=40 dan grid=10.
\end{eulercomment}
\begin{eulerprompt}
>plot3d("x^2+y^2"):
\end{eulerprompt}
\eulerimg{17}{images/EMT4Plot3D_Icha Nur Oktaviani Hartono_23030630027-003.png}
\begin{eulercomment}
Interaksi pengguna dapat dilakukan dengan parameter \textgreater{}user. Pengguna
dapat menekan tombol berikut ini.

- kiri, kanan, atas, bawah: memutar sudut pandang\\
- +,-: memperbesar atau memperkecil\\
- a: menghasilkan anaglyph (lihat di bawah)\\
- l: beralih memutar sumber cahaya (lihat di bawah)\\
- spasi: mengatur ulang ke default\\
- kembali: mengakhiri interaksi
\end{eulercomment}
\begin{eulerprompt}
>plot3d("exp(-x^2+y^2)",>user, ...
>  title="Turn with the vector keys (press return to finish)"):
\end{eulerprompt}
\eulerimg{17}{images/EMT4Plot3D_Icha Nur Oktaviani Hartono_23030630027-004.png}
\begin{eulercomment}
Rentang plot untuk fungsi dapat ditentukan dengan\\
- a, b: rentang x\\
- c, d: rentang y\\
- r: bujur sangkar simetris di sekitar (0,0).\\
- n: jumlah subinterval untuk plot.

Terdapat beberapa parameter untuk menskalakan fungsi atau mengubah
tampilan grafik.

fscale: skala untuk nilai fungsi (standarnya adalah \textless{}fscale).\\
scale: angka atau vektor 1x2 untuk menskalakan ke arah x dan y.\\
frame: jenis bingkai (default 1).
\end{eulercomment}
\begin{eulerprompt}
>plot3d("exp(-(x^2+y^2)/5)",r=10,n=80,fscale=4,scale=1.2,frame=3,>user):
\end{eulerprompt}
\eulerimg{17}{images/EMT4Plot3D_Icha Nur Oktaviani Hartono_23030630027-005.png}
\begin{eulercomment}
Tampilan dapat diubah dengan berbagai cara.\\
- distance: jarak pandang ke plot\\
- zoom: nilai zoom\\
- angle: sudut ke sumbu y negatif dalam radian\\
- height: ketinggian tampilan dalam radian\\
Nilai default dapat diperiksa atau diubah dengan fungsi view(). Fungsi
ini mengembalikan parameter dalam urutan di atas.
\end{eulercomment}
\begin{eulerprompt}
>view
\end{eulerprompt}
\begin{euleroutput}
  [5,  2.6,  2,  0.4]
\end{euleroutput}
\begin{eulercomment}
Jarak yang lebih dekat membutuhkan zoom yang lebih sedikit. Efeknya
lebih seperti lensa sudut lebar.

Dalam contoh berikut, angle=10 dan height=0 terlihat dari sumbu y
negatif. Label sumbu untuk y disembunyikan dalam kasus ini.
\end{eulercomment}
\begin{eulerprompt}
>plot3d("x^2+y",distance=3,zoom=1,angle=pi/2,height=0):
\end{eulerprompt}
\eulerimg{17}{images/EMT4Plot3D_Icha Nur Oktaviani Hartono_23030630027-006.png}
\begin{eulercomment}
Plot selalu terlihat ke bagian tengah kubus plot. Kalian dapat
memindahkan bagian tengah dengan parameter center.
\end{eulercomment}
\begin{eulerprompt}
>plot3d("x^4+y^2",a=0,b=1,c=-1,d=1,angle=-20°,height=20°, ...
>  center=[0.4,0,0],zoom=5):
\end{eulerprompt}
\eulerimg{17}{images/EMT4Plot3D_Icha Nur Oktaviani Hartono_23030630027-007.png}
\begin{eulercomment}
Plot sudah diatur secara default agar sesuai dengan kubus satuan untuk
dilihat. Jadi kalian tidak perlu mengunah distance atau zoomnya untuk
menyesuaikan dengan ukuran plot. Namun, Labels merujuk pada ukuran
sebenarnya.\\
Jika kalian mematikan perintah ini dengan scale=false, kalian perlu
meninjaklanjutinya, supaya plot yang kalian buat tetap sesuai dengan
plotting window, dengan mengubah jarak pandang (viewing distance) atau
zoom, dan memindahkan pusatnya.
\end{eulercomment}
\begin{eulerprompt}
>plot3d("5*exp(-x^2-y^2)",r=2,<fscale,<scale,distance=13,height=50°, ...
>  center=[0,0,-2],frame=3):
\end{eulerprompt}
\eulerimg{17}{images/EMT4Plot3D_Icha Nur Oktaviani Hartono_23030630027-008.png}
\begin{eulercomment}
Plot polar juga tersedia. Parameter polar=true digunakan untuk
menggambar plot polar. Fungsinya harus masih dalam fungsi x dan y.
Parameter "fscale" mengatur fungsi dengan skalanya sendiri. Jika
tidak, fungsi tersebut akan diskalakan supaya sesuai dengan
kubus/cube.
\end{eulercomment}
\begin{eulerprompt}
>plot3d("1/(x^2+y^2+1)",r=5,>polar, ...
>fscale=2,>hue,n=100,zoom=4,>contour,color=blue):
\end{eulerprompt}
\eulerimg{17}{images/EMT4Plot3D_Icha Nur Oktaviani Hartono_23030630027-009.png}
\begin{eulerprompt}
>function f(r) := exp(-r/2)*cos(r); ...
>plot3d("f(x^2+y^2)",>polar,scale=[1,1,0.4],r=pi,frame=3,zoom=4):
\end{eulerprompt}
\eulerimg{17}{images/EMT4Plot3D_Icha Nur Oktaviani Hartono_23030630027-010.png}
\begin{eulercomment}
Parameter "rotate" digunakan untuk merotasikan fungsi di x di sekitar
x-axis.\\
- rotate=1: menggunakan x-axis\\
- rotate=2: menggunakan z-axis
\end{eulercomment}
\begin{eulerprompt}
>plot3d("x^2+1",a=-1,b=1,rotate=true,grid=5):
\end{eulerprompt}
\eulerimg{17}{images/EMT4Plot3D_Icha Nur Oktaviani Hartono_23030630027-011.png}
\begin{eulerprompt}
>plot3d("x^2+1",a=-1,b=1,rotate=2,grid=5):
\end{eulerprompt}
\eulerimg{17}{images/EMT4Plot3D_Icha Nur Oktaviani Hartono_23030630027-012.png}
\begin{eulerprompt}
>plot3d("sqrt(25-x^2)",a=0,b=5,rotate=1):
\end{eulerprompt}
\eulerimg{17}{images/EMT4Plot3D_Icha Nur Oktaviani Hartono_23030630027-013.png}
\begin{eulerprompt}
>plot3d("x*sin(x)",a=0,b=6pi,rotate=2):
\end{eulerprompt}
\eulerimg{17}{images/EMT4Plot3D_Icha Nur Oktaviani Hartono_23030630027-014.png}
\begin{eulercomment}
Berikut adalah plot dengan tiga fungsi.
\end{eulercomment}
\begin{eulerprompt}
>plot3d("x","x^2+y^2","y",r=2,zoom=3.5,frame=3):
\end{eulerprompt}
\eulerimg{17}{images/EMT4Plot3D_Icha Nur Oktaviani Hartono_23030630027-015.png}
\eulerheading{Plot Kontur}
\begin{eulercomment}
Untuk plot, Euler menambahkan garis grid atau garis kisi-kisi. Sebagai
gantinya, dimungkinkan untuk menggunakan garis level dan rona satu
warna atau spectral rona warna. Euler dapat menggambar ketinggian
fungsi dalam plot dengan menggunakakn bayangan. Dalam semua plot 3D
Euler dapat menghasilkan anaglyph merah/can.\\
- \textgreater{}hue: Untuk mengaktifkan shading atau bayangan, bukan wires\\
- \textgreater{}contour: Untuk memplot garis kontur secara otomatis ke dalam plot
yang dibuat\\
- level=...(or level): Vektor nilai untuk garis kontur.

Default untuk level adalah level="auto", yang mana nantinya perintah
ini akan mengkomputasikan beberapa garis dengan level yang berbeda
secara otomatis. Seperti yang terlihat dalam plot, level sebenarnya
adalah rentang level.

Gaya default dapat diubah. Untuk plot kontur berikut, kita akan
menggunakan grid yang lebih halus untuk 100x100 titik, skala fungsi
dan plot, dan menggunakan sudut penglihatan yang berbeda.
\end{eulercomment}
\begin{eulerprompt}
>plot3d("exp(-x^2-y^2)",r=2,n=100,level="thin", ...
> >contour,>spectral,fscale=1,scale=1.1,angle=45°,height=20°):
\end{eulerprompt}
\eulerimg{17}{images/EMT4Plot3D_Icha Nur Oktaviani Hartono_23030630027-016.png}
\begin{eulerprompt}
>plot3d("exp(x*y)",angle=100°,>contour,color=green):
\end{eulerprompt}
\eulerimg{17}{images/EMT4Plot3D_Icha Nur Oktaviani Hartono_23030630027-017.png}
\begin{eulercomment}
Untuk shading, defaultnya adalah menggunakan wana abu-abu. Tetapi
warna ini dapat diubah karena interval warna atau interval spectral
dapat digunakan.\\
- \textgreater{}spectral: Untuk skema default spectral\\
- color: Untuk menambahkan warna tetentu atau skema spectral tertentu\\
Untuk contoh berikut, kita menggunakan skema default spectral dan
memperkecil nilai n untuk mendapatkan tampilan yang lebih halus.
\end{eulercomment}
\begin{eulerprompt}
>plot3d("x^2+y^2",>spectral,>contour,n=100):
\end{eulerprompt}
\eulerimg{17}{images/EMT4Plot3D_Icha Nur Oktaviani Hartono_23030630027-018.png}
\begin{eulercomment}
Alih-alih garis level otomatis, kita akan mengatur nilai-nilai dari
garis level. Hal ini akan menghasilkan garis level yang lebih tipis
alih-alih interval level.
\end{eulercomment}
\begin{eulerprompt}
>plot3d("x^2-y^2",0,5,0,5,level=-1:0.1:1,color=redgreen):
\end{eulerprompt}
\eulerimg{17}{images/EMT4Plot3D_Icha Nur Oktaviani Hartono_23030630027-019.png}
\begin{eulercomment}
Dalam plot berikut, kita akan menggunakan 2 level yang sangat luas
dari -0.1 sampai 1, dan dari 0.9 sampai 1. Level berikut akan
dimasukkan sebagai matriks dengan batas level sebagai kolom.\\
Selain itu, kita membentangkan grid dengan 10 interval dalam setiap
arah.
\end{eulercomment}
\begin{eulerprompt}
>plot3d("x^2+y^3",level=[-0.1,0.9;0,1], ...
>  >spectral,angle=30°,grid=10,contourcolor=gray):
\end{eulerprompt}
\eulerimg{17}{images/EMT4Plot3D_Icha Nur Oktaviani Hartono_23030630027-020.png}
\begin{eulercomment}
Dalam contoh berikut, kita memplot set, dimana

\end{eulercomment}
\begin{eulerformula}
\[
f(x,y) = x^y-y^x = 0
\]
\end{eulerformula}
\begin{eulercomment}
Kita menggunakan garis tunggal tipis untuk garis levelnya.
\end{eulercomment}
\begin{eulerprompt}
>plot3d("x^y-y^x",level=0,a=0,b=6,c=0,d=6,contourcolor=red,n=100):
\end{eulerprompt}
\eulerimg{17}{images/EMT4Plot3D_Icha Nur Oktaviani Hartono_23030630027-022.png}
\begin{eulercomment}
Pada Euler, kita dapat menampilkan garis konturnya di bawah plot.
Warna dan jarak ke plot dapat diatur secara spesifik nilainya.
\end{eulercomment}
\begin{eulerprompt}
>plot3d("x^2+y^4",>cp,cpcolor=green,cpdelta=0.2):
\end{eulerprompt}
\eulerimg{17}{images/EMT4Plot3D_Icha Nur Oktaviani Hartono_23030630027-023.png}
\begin{eulercomment}
Berikut adalah beberapa contoh mengenai gaya. Kita selalu
menonaktifkan frame, dan menggunakan warna tertentu untuk plot dan
gridnya.
\end{eulercomment}
\begin{eulerprompt}
>figure(2,2); ...
>expr="y^3-x^2"; ...
>figure(1);  ...
>  plot3d(expr,<frame,>cp,cpcolor=spectral); ...
>figure(2);  ...
>  plot3d(expr,<frame,>spectral,grid=10,cp=2); ...
>figure(3);  ...
>  plot3d(expr,<frame,>contour,color=gray,nc=5,cp=3,cpcolor=greenred); ...
>figure(4);  ...
>  plot3d(expr,<frame,>hue,grid=10,>transparent,>cp,cpcolor=gray); ...
>figure(0):
\end{eulerprompt}
\eulerimg{17}{images/EMT4Plot3D_Icha Nur Oktaviani Hartono_23030630027-024.png}
\begin{eulercomment}
Ada beberapa skema spectral yang lain, yaitu yang dinomori dari 1
hingga 9. Tetapi kalian juga dapat menggunakan perintah color=value,
dimana value:\\
- spectral: interval dari biru ke merah\\
- white: untuk warna yang lebih halus\\
yellowblue,purplegreen,blueyellow,greenred\\
- blueyellow,greenpurple,yellowblue,redgreen
\end{eulercomment}
\begin{eulerprompt}
>figure(3,3); ...
>for i=1:9;  ...
>  figure(i); plot3d("x^2+y^2",spectral=i,>contour,>cp,<frame,zoom=4);  ...
>end; ...
>figure(0):
\end{eulerprompt}
\eulerimg{17}{images/EMT4Plot3D_Icha Nur Oktaviani Hartono_23030630027-025.png}
\begin{eulercomment}
Sumber cahaya dapat diubah dengan 1 dan dengan kursor kunci selama
interaksi pengguna. Hal itu juga dapat diatur dengan parameter.\\
- light: arah untuk cahaya\\
- amb: cahaya disekitar antara 0 dan 1\\
Ingat bahwa program tidak membuat beda sisi dari plot. Jika tidak ada
shadows atau bayangan, maka kalian perlu menggunakan Povray.
\end{eulercomment}
\begin{eulerprompt}
>plot3d("-x^2-y^2", ...
>  hue=true,light=[0,1,1],amb=0,user=true, ...
>  title="Press l and cursor keys (return to exit)"):
\end{eulerprompt}
\eulerimg{17}{images/EMT4Plot3D_Icha Nur Oktaviani Hartono_23030630027-026.png}
\begin{eulercomment}
Parameter warna mengubah warna permukaan. Warna garis level juga dapat
diubah.
\end{eulercomment}
\begin{eulerprompt}
>plot3d("-x^2-y^2",color=rgb(0.2,0.2,0),hue=true,frame=false, ...
>  zoom=3,contourcolor=red,level=-2:0.1:1,dl=0.01):
\end{eulerprompt}
\eulerimg{17}{images/EMT4Plot3D_Icha Nur Oktaviani Hartono_23030630027-027.png}
\begin{eulercomment}
color=0 akan memberikan efek warna spesial, yaitu efek warna pelangi.
\end{eulercomment}
\begin{eulerprompt}
>plot3d("x^2/(x^2+y^2+1)",color=0,hue=true,grid=10):
\end{eulerprompt}
\eulerimg{17}{images/EMT4Plot3D_Icha Nur Oktaviani Hartono_23030630027-028.png}
\begin{eulercomment}
Kita juga dapat mengatur permukaan plotnya supaya transaparan.
\end{eulercomment}
\begin{eulerprompt}
>plot3d("x^2+y^2",>transparent,grid=10,wirecolor=red):
\end{eulerprompt}
\eulerimg{17}{images/EMT4Plot3D_Icha Nur Oktaviani Hartono_23030630027-029.png}
\eulerheading{Plot Implisit}
\begin{eulercomment}
Dalam 3D juga terdapat fungsi implisit. Euler akan menghasilkan
potongan melalui objek. Fitur dari plot3d juga termasuk plot implisit.
Plot ini menampilkan set nol dari fungsi tiga variabel. Solusi dari\\
\end{eulercomment}
\begin{eulerformula}
\[
f(x,y,z)=0
\]
\end{eulerformula}
\begin{eulercomment}
dapat divisualisasikan dengan potongan yang sejajar dengan bidang xy,
bidang xz, dan bidang yz.\\
- implicit=1: artinya potongan akan sejajar dengan bidang yz\\
- implicit=2: artinya potongan akan sejajar dengan bidang xz\\
- implicit=4: artinya potongan akan sejajar dengan bidang xy\\
Tambahakn nilai ini jika kalian mau. Dalam contoh berikut kita akan
memplot\\
\end{eulercomment}
\begin{eulerformula}
\[
M={(x,y,z):x^{2}+y^{3}+zy=1}
\]
\end{eulerformula}
\begin{eulerprompt}
>plot3d("x^2+y^3+z*y-1",r=5,implicit=3):
\end{eulerprompt}
\eulerimg{17}{images/EMT4Plot3D_Icha Nur Oktaviani Hartono_23030630027-032.png}
\begin{eulerprompt}
>c=1; d=1;
>plot3d("((x^2+y^2-c^2)^2+(z^2-1)^2)*((y^2+z^2-c^2)^2+(x^2-1)^2)*((z^2+x^2-c^2)^2+(y^2-1)^2)-d",r=2,<frame,>implicit,>user): 
\end{eulerprompt}
\eulerimg{17}{images/EMT4Plot3D_Icha Nur Oktaviani Hartono_23030630027-033.png}
\begin{eulerprompt}
>plot3d("x^2+y^2+4*x*z+z^3",>implicit,r=2,zoom=2.5):
\end{eulerprompt}
\eulerimg{17}{images/EMT4Plot3D_Icha Nur Oktaviani Hartono_23030630027-034.png}
\eulerheading{Memplot data 3D}
\begin{eulercomment}
Sama seperti plot2d, plot3d juga dapat menerima data. Untuk objek 3D,
kalian perlu menyiapkan matriks nilai x,y,dan z, atau 3 fungsi atau
ekspresi fx(x,y),fy(x,y),fz(x,y).

\end{eulercomment}
\begin{eulerformula}
\[
\gamma(t,s) = (x(t,s),y(t,s),z(t,s))
\]
\end{eulerformula}
\begin{eulercomment}
Karena x,y,dan z adalah matriks, kita mengasumsikan bahwa (t,s)
dijalankan dengan grid persegi. Hasilnya, kita akan memplot gambar
dari segitiga

Kalian dapat menggunakan bahasa matriks dalam Euler untuk memproduksi
titik koordinat ini secara efektif.

Dalam contoh berikut, kita akan menggunakan cektor dari nilai t dan
kolom vektor dari nilai s untuk parameter permukaan dari bola. Saat
menggambar, kita dapat menandai bagian tertentu, dalam kasus kita,
kita menandai bagian polarnya.
\end{eulercomment}
\begin{eulerprompt}
>t=linspace(0,2pi,180); s=linspace(-pi/2,pi/2,90)'; ...
>x=cos(s)*cos(t); y=cos(s)*sin(t); z=sin(s); ...
>plot3d(x,y,z,>hue, ...
>color=blue,<frame,grid=[10,20], ...
>values=s,contourcolor=red,level=[90°-24°;90°-22°], ...
>scale=1.4,height=50°):
\end{eulerprompt}
\eulerimg{17}{images/EMT4Plot3D_Icha Nur Oktaviani Hartono_23030630027-036.png}
\begin{eulercomment}
Berikut adalah contoh, dimana grafiknya adalah sebuah fungsi.
\end{eulercomment}
\begin{eulerprompt}
>t=-1:0.1:1; s=(-1:0.1:1)'; plot3d(t,s,t*s,grid=10):
\end{eulerprompt}
\eulerimg{17}{images/EMT4Plot3D_Icha Nur Oktaviani Hartono_23030630027-037.png}
\begin{eulercomment}
Bagaimanapun, kita dapat membuat berbagai jenis permukaan. Berikut
adalah permukaan yang sama dengan fungsi

\end{eulercomment}
\begin{eulerformula}
\[
x = y \, z
\]
\end{eulerformula}
\begin{eulerprompt}
>plot3d(t*s,t,s,angle=180°,grid=10):
\end{eulerprompt}
\eulerimg{17}{images/EMT4Plot3D_Icha Nur Oktaviani Hartono_23030630027-039.png}
\begin{eulercomment}
Dengan usaha yang lebih, kita dapat membuat berbagai jenis permukaan.

Dalam contoh berikut kita akan membuat tampilan bayangan dari bola
berubah. Koordinat bola biasanya adalah

\end{eulercomment}
\begin{eulerformula}
\[
\gamma(t,s) = (\cos(t)\cos(s),\sin(t)\sin(s),\cos(s))
\]
\end{eulerformula}
\begin{eulercomment}
dengan

\end{eulercomment}
\begin{eulerformula}
\[
0 \le t \le 2\pi, \quad \frac{-\pi}{2} \le s \le \frac{\pi}{2}.
\]
\end{eulerformula}
\begin{eulercomment}
Kita akan mengubah ini dengan faktor

\end{eulercomment}
\begin{eulerformula}
\[
d(t,s) = \frac{\cos(4t)+\cos(8s)}{4}.
\]
\end{eulerformula}
\begin{eulerprompt}
>t=linspace(0,2pi,320); s=linspace(-pi/2,pi/2,160)'; ...
>d=1+0.2*(cos(4*t)+cos(8*s)); ...
>plot3d(cos(t)*cos(s)*d,sin(t)*cos(s)*d,sin(s)*d,hue=1, ...
>  light=[1,0,1],frame=0,zoom=5):
\end{eulerprompt}
\eulerimg{17}{images/EMT4Plot3D_Icha Nur Oktaviani Hartono_23030630027-043.png}
\begin{eulercomment}
Tentu saja titik awan juga memungkinkan. Untuk memplot titik data di
ruang, kita membutuhkan tiga vektor untuk koordinat titikya.\\
Untuk mengaturnya masih sama dengan pada plot2d, yaitu dengan
points=true;
\end{eulercomment}
\begin{eulerprompt}
>n=500;  ...
>  plot3d(normal(1,n),normal(1,n),normal(1,n),points=true,style="."):
\end{eulerprompt}
\eulerimg{17}{images/EMT4Plot3D_Icha Nur Oktaviani Hartono_23030630027-044.png}
\begin{eulercomment}
Kita juga dapa memplot kurva di 3D. Dalam kasus ini, akan lebih mudah
untuk mengkomputasikan terlebih dahulu titik dari kurvanya. Untuk
kurva dalam bidang, kita akan menggunakan deretan dari koordinat dan
parameter wire=true.
\end{eulercomment}
\begin{eulerprompt}
>t=linspace(0,8pi,500); ...
>plot3d(sin(t),cos(t),t/10,>wire,zoom=3):
\end{eulerprompt}
\eulerimg{17}{images/EMT4Plot3D_Icha Nur Oktaviani Hartono_23030630027-045.png}
\begin{eulerprompt}
>t=linspace(0,4pi,1000); plot3d(cos(t),sin(t),t/2pi,>wire, ...
>linewidth=3,wirecolor=blue):
\end{eulerprompt}
\eulerimg{17}{images/EMT4Plot3D_Icha Nur Oktaviani Hartono_23030630027-046.png}
\begin{eulerprompt}
>X=cumsum(normal(3,100)); ...
> plot3d(X[1],X[2],X[3],>anaglyph,>wire):
\end{eulerprompt}
\eulerimg{17}{images/EMT4Plot3D_Icha Nur Oktaviani Hartono_23030630027-047.png}
\begin{eulercomment}
EMT juga dapat memplot dalam anaglyph mode. Untuk melihat plot mode
anaglyph, kalian membutuhkan kacamata red/cyan.
\end{eulercomment}
\begin{eulerprompt}
> plot3d("x^2+y^3",>anaglyph,>contour,angle=30°):
\end{eulerprompt}
\eulerimg{17}{images/EMT4Plot3D_Icha Nur Oktaviani Hartono_23030630027-048.png}
\begin{eulercomment}
Seringkali, skema warna spectral digunakan untuk plot. Hal ini
menekankan ketinggian dari fungsinya.
\end{eulercomment}
\begin{eulerprompt}
>plot3d("x^2*y^3-y",>spectral,>contour,zoom=3.2):
\end{eulerprompt}
\eulerimg{17}{images/EMT4Plot3D_Icha Nur Oktaviani Hartono_23030630027-049.png}
\begin{eulercomment}
Euler juga dapat memplot parameter permukaan, saat parameternya adalah
nilai x,y,dan z dari sebuah gambar dalam bentuk grid persegi panjang
di ruang 3D.

Untuk ocntoh berikut, kita mengeset parameter u dan v, dan
menghasilkan koordinat ruang dari parameter tadi.
\end{eulercomment}
\begin{eulerprompt}
>u=linspace(-1,1,10); v=linspace(0,2*pi,50)'; ...
>X=(3+u*cos(v/2))*cos(v); Y=(3+u*cos(v/2))*sin(v); Z=u*sin(v/2); ...
>plot3d(X,Y,Z,>anaglyph,<frame,>wire,scale=2.3):
\end{eulerprompt}
\eulerimg{17}{images/EMT4Plot3D_Icha Nur Oktaviani Hartono_23030630027-050.png}
\begin{eulercomment}
Berikut adalah contoh yang lebih rumit, dengan kacamata rec/cyan yang
majestik.
\end{eulercomment}
\begin{eulerprompt}
>u:=linspace(-pi,pi,160); v:=linspace(-pi,pi,400)';  ...
>x:=(4*(1+.25*sin(3*v))+cos(u))*cos(2*v); ...
>y:=(4*(1+.25*sin(3*v))+cos(u))*sin(2*v); ...
> z=sin(u)+2*cos(3*v); ...
>plot3d(x,y,z,frame=0,scale=1.5,hue=1,light=[1,0,-1],zoom=2.8,>anaglyph):
\end{eulerprompt}
\eulerimg{17}{images/EMT4Plot3D_Icha Nur Oktaviani Hartono_23030630027-051.png}
\eulerheading{Plot Statistik}
\begin{eulercomment}
Bar plot juga dapat dihasilkan menggunakan EMT. Untuk membuatnya, kita
perlu menyediakan

- x: vektor baris dengan n+1 elemen\\
- y: vektor kolom dengan n+1 elemen\\
- z: matriks nxn

z dapat lebih besar, namun hanya nilai nxn yang akan digunakan.

Dalam contoh berikut, kita lebih dulu mengkomputasi nilai. Lalu kita
menyesuaikan x dan y sehingga pusat vektor dalam nilai digunakan.
\end{eulercomment}
\begin{eulerprompt}
>x=-1:0.1:1; y=x'; z=x^2+y^2; ...
>xa=(x|1.1)-0.05; ya=(y_1.1)-0.05; ...
>plot3d(xa,ya,z,bar=true):
\end{eulerprompt}
\eulerimg{17}{images/EMT4Plot3D_Icha Nur Oktaviani Hartono_23030630027-052.png}
\begin{eulercomment}
Kita juga dapat membagi permukaan plotnya dalam dua atau lebih bagian.
\end{eulercomment}
\begin{eulerprompt}
>x=-1:0.1:1; y=x'; z=x+y; d=zeros(size(x)); ...
>plot3d(x,y,z,disconnect=2:2:20):
\end{eulerprompt}
\eulerimg{17}{images/EMT4Plot3D_Icha Nur Oktaviani Hartono_23030630027-053.png}
\begin{eulercomment}
Jika memuat atau menghasilkan matriks data M dari file dan perlu
memplotnya dalam 3D, kalian dapat mengatur skala matriks ke [-1,1]
dengan scale(M), atau mengatur skala matriks dengan \textgreater{}zscale. Hal ini
dapat dikombinasikan dengan faktor penskalaan individual yang
diterapkan sebagai tambahan.
\end{eulercomment}
\begin{eulerprompt}
>i=1:20; j=i'; ...
>plot3d(i*j^2+100*normal(20,20),>zscale,scale=[1,1,1.5],angle=-40°,zoom=1.8):
\end{eulerprompt}
\eulerimg{17}{images/EMT4Plot3D_Icha Nur Oktaviani Hartono_23030630027-054.png}
\begin{eulerprompt}
>Z=intrandom(5,100,6); v=zeros(5,6); ...
>loop 1 to 5; v[#]=getmultiplicities(1:6,Z[#]); end; ...
>columnsplot3d(v',scols=1:5,ccols=[1:5]):
\end{eulerprompt}
\eulerimg{17}{images/EMT4Plot3D_Icha Nur Oktaviani Hartono_23030630027-055.png}
\eulerheading{Permukaan Benda Putar}
\begin{eulerprompt}
>plot2d("(x^2+y^2-1)^3-x^2*y^3",r=1.3, ...
>style="#",color=red,<outline, ...
>level=[-2;0],n=100):
\end{eulerprompt}
\eulerimg{17}{images/EMT4Plot3D_Icha Nur Oktaviani Hartono_23030630027-056.png}
\begin{eulerprompt}
>ekspresi &= (x^2+y^2-1)^3-x^2*y^3; $ekspresi
\end{eulerprompt}
\begin{eulerformula}
\[
\left(y^2+x^2-1\right)^3-x^2\,y^3
\]
\end{eulerformula}
\begin{eulercomment}
Kita ingin memutar kurva bentuk hati tersebut di sekitar sumbu y.
Berikut adalah ekspresinya, yang mendefinisikan kurva bentuk hati:

\end{eulercomment}
\begin{eulerformula}
\[
f(x,y)=(x^2+y^2-1)^3-x^2.y^3.
\]
\end{eulerformula}
\begin{eulercomment}
Selanjutnya kita mengatur:

\end{eulercomment}
\begin{eulerformula}
\[
x=r.cos(a),\quad y=r.sin(a).
\]
\end{eulerformula}
\begin{eulerprompt}
>function fr(r,a) &= ekspresi with [x=r*cos(a),y=r*sin(a)] | trigreduce; $fr(r,a)
\end{eulerprompt}
\begin{eulerformula}
\[
\left(r^2-1\right)^3+\frac{\left(\sin \left(5\,a\right)-\sin \left(  3\,a\right)-2\,\sin a\right)\,r^5}{16}
\]
\end{eulerformula}
\begin{eulercomment}
Hal ini berlaku juga untuk fungsi numerik, yang menyelesaikan r jika
diberikan a. Dengan fungsi tersebut kita dapat memplot bentuk hati
yang akan diputar sebagai permukaan parametrik.
\end{eulercomment}
\begin{eulerprompt}
>function map f(a) := bisect("fr",0,2;a); ...
>t=linspace(-pi/2,pi/2,100); r=f(t);  ...
>s=linspace(pi,2pi,100)'; ...
>plot3d(r*cos(t)*sin(s),r*cos(t)*cos(s),r*sin(t), ...
>>hue,<frame,color=red,zoom=4,amb=0,max=0.7,grid=12,height=50°):
\end{eulerprompt}
\eulerimg{17}{images/EMT4Plot3D_Icha Nur Oktaviani Hartono_23030630027-061.png}
\begin{eulercomment}
Berikut adalah plot 3D dari bentuk di atas, yang diputar mengelilingi
sumbu z. Kita mendefinisikan fungsinya, yang mendeskripsikan objek
tersebut.
\end{eulercomment}
\begin{eulerprompt}
>function f(x,y,z) ...
\end{eulerprompt}
\begin{eulerudf}
  r=x^2+y^2;
  return (r+z^2-1)^3-r*z^3;
   endfunction
\end{eulerudf}
\begin{eulerprompt}
>plot3d("f(x,y,z)", ...
>xmin=0,xmax=1.2,ymin=-1.2,ymax=1.2,zmin=-1.2,zmax=1.4, ...
>implicit=1,angle=-30°,zoom=2.5,n=[10,100,60],>anaglyph):
\end{eulerprompt}
\eulerimg{17}{images/EMT4Plot3D_Icha Nur Oktaviani Hartono_23030630027-062.png}
\eulerheading{Plot 3D Khusus}
\begin{eulercomment}
Fungsi plot3d memang bagus, tapi fungsi tersebut belum dapat memenuhi
semua yang kita butuhkan. Selain hal-hal dasaar, fungsi tersebut
memungkinkan untuk menambahkan frame pada semua objek yang kalian
suka.

Meskipun Euler bukan program 3D, dia dapat menggabungkankan beberapa
objek dasar. Kita akan mencoba untuk memvisualisasikan paraboloid dan
garis singgungnya.
\end{eulercomment}
\begin{eulerprompt}
>function myplot ...
\end{eulerprompt}
\begin{eulerudf}
    y=-1:0.01:1; x=(-1:0.01:1)';
    plot3d(x,y,0.2*(x-0.1)/2,<scale,<frame,>hue, ..
      hues=0.5,>contour,color=orange);
    h=holding(1);
    plot3d(x,y,(x^2+y^2)/2,<scale,<frame,>contour,>hue);
    holding(h);
  endfunction
\end{eulerudf}
\begin{eulercomment}
Fungsi framedplot() memungkinkan kita untuk menambahkan frame dan
mengatur viewsnya.
\end{eulercomment}
\begin{eulerprompt}
>framedplot("myplot",[-1,1,-1,1,0,1],height=0,angle=-30°, ...
>  center=[0,0,-0.7],zoom=3):
\end{eulerprompt}
\eulerimg{17}{images/EMT4Plot3D_Icha Nur Oktaviani Hartono_23030630027-063.png}
\begin{eulercomment}
Dengan cara yang sama, kalian dapat memplot bidnag konturnya secara
manual. Ingat bahwa plot3d() mengatur jendela kek fullwindow() secara
default, tetapi plotcontourplane() mengasumsikannya.
\end{eulercomment}
\begin{eulerprompt}
>x=-1:0.02:1.1; y=x'; z=x^2-y^4;
>function myplot (x,y,z) ...
\end{eulerprompt}
\begin{eulerudf}
    zoom(2);
    wi=fullwindow();
    plotcontourplane(x,y,z,level="auto",<scale);
    plot3d(x,y,z,>hue,<scale,>add,color=white,level="thin");
    window(wi);
    reset();
  endfunction
\end{eulerudf}
\begin{eulerprompt}
>myplot(x,y,z):
\end{eulerprompt}
\eulerimg{27}{images/EMT4Plot3D_Icha Nur Oktaviani Hartono_23030630027-064.png}
\eulerheading{Animasi}
\begin{eulercomment}
Euler dapat menggunakan frame untuk memperkirakan animasi.

Salah satu fungsinya yang menggunakan teknik ini adalah rotasi. Fungsi
ini dapat mengubah sudut pandang adn menggambar ulang plot 3D. Fungsi
ini memanggil addpage() untuk setiap plot baru. Akhirnya fungsi ini
menganimasikan plot tersebut.

Silakan pelajari sumber rotate untuk melihat lebih detail.
\end{eulercomment}
\begin{eulerprompt}
>function testplot () := plot3d("x^2+y^3"); ...
>rotate("testplot"); testplot():
\end{eulerprompt}
\begin{euleroutput}
  Press space to stop, return to end
\end{euleroutput}
\eulerimg{27}{images/EMT4Plot3D_Icha Nur Oktaviani Hartono_23030630027-065.png}
\eulerheading{Menggambar Povray}
\begin{eulercomment}
Dengan bantuan file Euler povray, Euler dapat menghasilkan file
Povray. Hasilkan akan sangat indah untuk dipandang.

Kalian perlu menginstall Povray (32bit atau 64bit) dari
http://www.povray.org/, dan meletakkan sub direktori "bin" dari Povray ke dalam environment, atau mengatur variabel "defaultpovray" dengan jalur lengkap yang mengarah ke "pvengine.exe".

Antarmuka Povray dari Euler menghasilkan file Povray du deirektori
home pengguna, dan memanggil Povray untuk mengurai file-file ini. Nama
file default adalah current.pov, dan direktori defaultnya adalah
eulerhome(), biasanya c:\textbackslash{}Users\textbackslash{}Username\textbackslash{}Euler. Povray menghasilkan
sebuah file PNG, yang dapat dimuat oleh Euler ke dalam notebook. Untuk
membersihkan berkas-berkas ini, gunakan povclear().

Fungsi pov3d memiliki semangat yang sama dengan plot3d. Fungsi ini
dapat menghasilkan grafik dari sebuah fungsi f(x,y), atau sebuah
permukaan dengan koordinat X,Y,Z dalam bentuk matriks, termasuk
garis-garis level yang bersifat opsional. Fungsi ini memulai raytracer
secara otomatis, dan memuat adegan ke dalam notebook Euler.

Selain pov3d(), ada banyak fungsi yang menghasilkan objek Povray.
Fungsi-fungsi ini mengembalikan string, yang berisi kode Povray untuk
objek. Untuk menggunakan fungsi-fungsi ini, mulai file Povray dengan
povstart(). Kemudian gunakan writeln(...) untuk menulis objek ke file
scene. Terakhir, akhiri file dengan povend(). Secara default,
raytracer akan dimulai, dan PNG akan dimasukkan ke dalam buku catatan
Euler.

Fungsi objek memiliki parameter yang disebut “look”, yang membutuhkan
string dengan kode povray untuk tekstur dan hasil akhir objek. Fungsi
povlook() dapat digunakan untuk menghasilkan string ini. Fungsi ini
memiliki parameter untuk warna, transparansi, Phong Shading, dll.

Perhatikan bahwa Povray universe memiliki sistem koordinat lain.
Antarmuka ini menerjemahkan semua koordinat ke sistem Povray. Jadi
Anda dapat tetap berpikir dalam sistem koordinat Euler dengan z yang
mengarah vertikal ke atas, dan sumbu x, y, z di tangan kanan.\\
nda perlu memuat file povray.\\
Translated with DeepL.com (free version)
\end{eulercomment}
\begin{eulerprompt}
>load povray;
\end{eulerprompt}
\begin{eulercomment}
Pastikakn direktori bin Povray berada di jalur yang benar. Jika tidak,
edit variabel berikut sehingga berisi jalur ke eksekusi povray.
\end{eulercomment}
\begin{eulerprompt}
>defaultpovray="C:\(\backslash\)Program Files\(\backslash\)POV-Ray\(\backslash\)v3.7\(\backslash\)bin\(\backslash\)pvengine.exe"
\end{eulerprompt}
\begin{euleroutput}
  C:\(\backslash\)Program Files\(\backslash\)POV-Ray\(\backslash\)v3.7\(\backslash\)bin\(\backslash\)pvengine.exe
\end{euleroutput}
\begin{eulercomment}
Untuk kesan pertama, kita akan memplot fungsi yang sederhana. Perintah
berikut akan menghasilkan file povray di direktori user kalian, dan
menjalankan Povray untuk melacak sinar pada file ini.

Jika kalian menjalankakn perintah berikut, Povray GUI seharusnya
terbuka, menjalankan file, dan tertutup secara otomatis. Karena hal
keamanan, kalian akan ditanya, jika kaliani ingin mengizinkan file exe
untuk dijalankan. Kalian dapat mengklik cancel jika ingin menghentikan
pertanyaan berikutnya. Kalian perlu mengklik OK di jendela Povray
untuk mengetahu dialog awal Povray.
\end{eulercomment}
\begin{eulerprompt}
>plot3d("x^2+y^2",zoom=2):
\end{eulerprompt}
\eulerimg{27}{images/EMT4Plot3D_Icha Nur Oktaviani Hartono_23030630027-066.png}
\begin{eulerprompt}
>pov3d("x^2+y^2",zoom=3);
\end{eulerprompt}
\eulerimg{27}{images/EMT4Plot3D_Icha Nur Oktaviani Hartono_23030630027-067.png}
\begin{eulercomment}
Kita dapat membuat fungsi menjadi transparan dan menambahkan beberapa
hal. Kita juga dapat menambahkan level garis untuk plot fungsi.
\end{eulercomment}
\begin{eulerprompt}
>pov3d("x^2+y^3",axiscolor=red,angle=-45°,>anaglyph, ...
>  look=povlook(cyan,0.2),level=-1:0.5:1,zoom=3.8);
\end{eulerprompt}
\eulerimg{27}{images/EMT4Plot3D_Icha Nur Oktaviani Hartono_23030630027-068.png}
\begin{eulercomment}
Terkadang kita perlu untuk mencegah penskalaan fungsi, dan menskalakan
fungsi dengan tangan.

Kita memplot kumpulan titik pada bidang kompleks, dimana hasil kali
jarak ke 1 dan -1 sama dengan 1.
\end{eulercomment}
\begin{eulerprompt}
>pov3d("((x-1)^2+y^2)*((x+1)^2+y^2)/40",r=2, ...
>  angle=-120°,level=1/40,dlevel=0.005,light=[-1,1,1],height=10°,n=50, ...
>  <fscale,zoom=3.8);
\end{eulerprompt}
\eulerimg{27}{images/EMT4Plot3D_Icha Nur Oktaviani Hartono_23030630027-069.png}
\eulerheading{Memplot dengan Koordinat}
\begin{eulercomment}
Alih-alih dengan fungsi, kita dapat memplot dengan koordinat. Seperti
dalam plot3d, kita membutuhkan tiga matriks untuk mendefinisikakn
suatu objek.

Sebagai contoh kita memutar fungsi pada sumbu z.
\end{eulercomment}
\begin{eulerprompt}
>function f(x) := x^3-x+1; ...
>x=-1:0.01:1; t=linspace(0,2pi,50)'; ...
>Z=x; X=cos(t)*f(x); Y=sin(t)*f(x); ...
>pov3d(X,Y,Z,angle=40°,look=povlook(red,0.1),height=50°,axis=0,zoom=4,light=[10,5,15]);
\end{eulerprompt}
\eulerimg{27}{images/EMT4Plot3D_Icha Nur Oktaviani Hartono_23030630027-070.png}
\begin{eulercomment}
Pda contoh berikut, kita memplot gelombang teredam. Kita menghasilkan
gelombang dengan bahasa matriks Euler.

Kita juga menunjukkan bagaimana objek tambahan dapat ditambahkan ke
adegan pov3d. Untuk menghasilkan objeknya, perhatikan contoh berikut.
Ingat bahwa plot3d mengatur plot sehingga plot yang dihasilkan akan
sesuai dengan kubus satuan.
\end{eulercomment}
\begin{eulerprompt}
>r=linspace(0,1,80); phi=linspace(0,2pi,80)'; ...
>x=r*cos(phi); y=r*sin(phi); z=exp(-5*r)*cos(8*pi*r)/3;  ...
>pov3d(x,y,z,zoom=6,axis=0,height=30°,add=povsphere([0.5,0,0.25],0.15,povlook(red)), ...
>  w=500,h=300);
\end{eulerprompt}
\eulerimg{16}{images/EMT4Plot3D_Icha Nur Oktaviani Hartono_23030630027-071.png}
\begin{eulercomment}
Dengan metode bayangan canggih Povray, hanya sedikit titik yang bisa
menghasilkan permukaan yang sangat halus. Hanya pada batas-batas dan
bayangan, trik ini bisa terlihat jelas.

Untuk ini, kita perlu menambahkan vektor normal di setiap titik
matriks.
\end{eulercomment}
\begin{eulerprompt}
>Z &= x^2*y^3
\end{eulerprompt}
\begin{euleroutput}
  
                                   2  3
                                  x  y
  
\end{euleroutput}
\begin{eulercomment}
Persamaan untuk permukaannya adalah [x,y,z]. Kita akan menghitung dua
turunan terhadap x dan y dari persamaan ini lalu mengambil hasil cross
produknya sebagai normal.
\end{eulercomment}
\begin{eulerprompt}
>dx &= diff([x,y,Z],x); dy &= diff([x,y,Z],y);
\end{eulerprompt}
\begin{eulercomment}
Kita mendefinisikan vektor normal sebagai cross produk dari turunan
ini, dan mendefinisikan koordinat fungsinya.
\end{eulercomment}
\begin{eulerprompt}
>N &= crossproduct(dx,dy); NX &= N[1]; NY &= N[2]; NZ &= N[3]; N,
\end{eulerprompt}
\begin{euleroutput}
  
                                 3       2  2
                         [- 2 x y , - 3 x  y , 1]
  
\end{euleroutput}
\begin{eulercomment}
Kita hanya akan menggunakan 25 titik.
\end{eulercomment}
\begin{eulerprompt}
>x=-1:0.5:1; y=x';
>pov3d(x,y,Z(x,y),angle=10°, ...
>  xv=NX(x,y),yv=NY(x,y),zv=NZ(x,y),<shadow);
\end{eulerprompt}
\eulerimg{27}{images/EMT4Plot3D_Icha Nur Oktaviani Hartono_23030630027-072.png}
\begin{eulercomment}
Berikut ini adalah simpul Trefoil yang dibuat oleh A. Busser di
Povray. Ada versi yang lebih baik dai ini dalam contoh.

See: Examples\textbackslash{}Trefoil Knot \textbar{} Trefoil Knot

Untuk tampilan yang bagus dengan tidak terlalu banayk titik, kita
tambahkan vektor normal di sini. Kita menggunakan Maxima untuk
menghitung normal untuk kita. Pertama, tiga fungsi untuk koordinat
sebagai ekspresi simbolik.
\end{eulercomment}
\begin{eulerprompt}
>X &= ((4+sin(3*y))+cos(x))*cos(2*y); ...
>Y &= ((4+sin(3*y))+cos(x))*sin(2*y); ...
>Z &= sin(x)+2*cos(3*y);
\end{eulerprompt}
\begin{eulercomment}
Kemudian dua turunan vektor untuk x dan y.
\end{eulercomment}
\begin{eulerprompt}
>dx &= diff([X,Y,Z],x); dy &= diff([X,Y,Z],y);
\end{eulerprompt}
\begin{eulercomment}
Selanjutnya normal, yang merupakan corss produk dari dua turunan.
\end{eulercomment}
\begin{eulerprompt}
>dn &= crossproduct(dx,dy);
\end{eulerprompt}
\begin{eulercomment}
Selanjutny kita menghitung semua tadi secara numerik.
\end{eulercomment}
\begin{eulerprompt}
>x:=linspace(-%pi,%pi,40); y:=linspace(-%pi,%pi,100)';
\end{eulerprompt}
\begin{eulercomment}
Vektor normal adalah evaluasi dari ekspresi simbolik dn[i] untuk
i=1,2,3. Sintaks untuk ini adalah \&"ekspresi"(parameter). Ini adalah
sebuah alternatif dari metode pada contoh sebelumnya, dimana kita
mendefinisikan ekspresi simbolik NX, NY, NZ terlebih dahulu.
\end{eulercomment}
\begin{eulerprompt}
>pov3d(X(x,y),Y(x,y),Z(x,y),>anaglyph,axis=0,zoom=5,w=450,h=350, ...
>  <shadow,look=povlook(blue), ...
>  xv=&"dn[1]"(x,y), yv=&"dn[2]"(x,y), zv=&"dn[3]"(x,y));
\end{eulerprompt}
\eulerimg{21}{images/EMT4Plot3D_Icha Nur Oktaviani Hartono_23030630027-073.png}
\begin{eulercomment}
Kita juga dapat menghasilkan grid di 3D.
\end{eulercomment}
\begin{eulerprompt}
>povstart(zoom=4); ...
>x=-1:0.5:1; r=1-(x+1)^2/6; ...
>t=(0°:30°:360°)'; y=r*cos(t); z=r*sin(t); ...
>writeln(povgrid(x,y,z,d=0.02,dballs=0.05)); ...
>povend();
\end{eulerprompt}
\eulerimg{27}{images/EMT4Plot3D_Icha Nur Oktaviani Hartono_23030630027-074.png}
\begin{eulercomment}
Dengan povgrid(), kurva adalah hal yang mungkin.
\end{eulercomment}
\begin{eulerprompt}
>povstart(center=[0,0,1],zoom=3.6); ...
>t=linspace(0,2,1000); r=exp(-t); ...
>x=cos(2*pi*10*t)*r; y=sin(2*pi*10*t)*r; z=t; ...
>writeln(povgrid(x,y,z,povlook(red))); ...
>writeAxis(0,2,axis=3); ...
>povend();
\end{eulerprompt}
\eulerimg{27}{images/EMT4Plot3D_Icha Nur Oktaviani Hartono_23030630027-075.png}
\eulerheading{Povray Objek}
\begin{eulercomment}
Di atas, kita sudah menggunakan pov3d untuk memplot permukaan.
Antarmuka Povray di Euler juga dapat menghasilkan objek Povray.
Objek-objek ini disimpan sebagai string di Euler, dan perlu ditulis ke
file Povray.

Kita memulai output dengan povstart().
\end{eulercomment}
\begin{eulerprompt}
>povstart(zoom=4);
\end{eulerprompt}
\begin{eulercomment}
Pertama kita definisikan tiga silinder, dan menyimpannya di string
pada Euler.

Fungsi povx() etc. hanyak enghasilkan vektor [1,0,0], yang dapat
digunakan sebagai gantinya.
\end{eulercomment}
\begin{eulerprompt}
>c1=povcylinder(-povx,povx,1,povlook(red)); ...
>c2=povcylinder(-povy,povy,1,povlook(yellow)); ...
>c3=povcylinder(-povz,povz,1,povlook(blue)); ...
\end{eulerprompt}
\begin{eulercomment}
String tersebut memuat kode Povray, yang tidak perlu kita pahami pada
saat itu.
\end{eulercomment}
\begin{eulerprompt}
>c2
\end{eulerprompt}
\begin{euleroutput}
  cylinder \{ <0,0,-1>, <0,0,1>, 1
   texture \{ pigment \{ color rgb <0.941176,0.941176,0.392157> \}  \} 
   finish \{ ambient 0.2 \} 
   \}
\end{euleroutput}
\begin{eulercomment}
Seperti yang terlihat, kita menambahkan tekstur ke dalam objek dengan
3 warna yang berbeda.

Hal ini dilakukan dengan povlook(), yang akan mengembalikan string
dengan kode Povray yang relevan. Kita dapat menggunakan warna default
Euler, atau mendefinisikan warna kita sendiri. Kita juga dapat
menambahkan transparansi, atau mengubah cahaya sekitar.
\end{eulercomment}
\begin{eulerprompt}
>povlook(rgb(0.1,0.2,0.3),0.1,0.5)
\end{eulerprompt}
\begin{euleroutput}
   texture \{ pigment \{ color rgbf <0.101961,0.2,0.301961,0.1> \}  \} 
   finish \{ ambient 0.5 \} 
  
\end{euleroutput}
\begin{eulercomment}
Sekarang kita mendefinisikan objek perpotonggan, dan menulis hasilnya
ke file.
\end{eulercomment}
\begin{eulerprompt}
>writeln(povintersection([c1,c2,c3]));
\end{eulerprompt}
\begin{eulercomment}
Perpotongan dari tiga silinder sulit untuk divisualisasikan, jika
kalian tidak pernah melihatnya sebelumnya.
\end{eulercomment}
\begin{eulerprompt}
>povend;
\end{eulerprompt}
\eulerimg{27}{images/EMT4Plot3D_Icha Nur Oktaviani Hartono_23030630027-076.png}
\begin{eulercomment}
Fungsi berikut menghasilkan fraktal secara rekursif.

Fungsi yang pertama memperlihatkan, bagaimana Euler mengatasi objek
sederhana Povray. Fungsi povbox() mengembalikan sebuah string, memuat
koordinat kotak, tekstur, dan hasil akhir.
\end{eulercomment}
\begin{eulerprompt}
>function onebox(x,y,z,d) := povbox([x,y,z],[x+d,y+d,z+d],povlook());
>function fractal (x,y,z,h,n) ...
\end{eulerprompt}
\begin{eulerudf}
   if n==1 then writeln(onebox(x,y,z,h));
   else
     h=h/3;
     fractal(x,y,z,h,n-1);
     fractal(x+2*h,y,z,h,n-1);
     fractal(x,y+2*h,z,h,n-1);
     fractal(x,y,z+2*h,h,n-1);
     fractal(x+2*h,y+2*h,z,h,n-1);
     fractal(x+2*h,y,z+2*h,h,n-1);
     fractal(x,y+2*h,z+2*h,h,n-1);
     fractal(x+2*h,y+2*h,z+2*h,h,n-1);
     fractal(x+h,y+h,z+h,h,n-1);
   endif;
  endfunction
\end{eulerudf}
\begin{eulerprompt}
>povstart(fade=10,<shadow);
>fractal(-1,-1,-1,2,4);
>povend();
\end{eulerprompt}
\eulerimg{27}{images/EMT4Plot3D_Icha Nur Oktaviani Hartono_23030630027-077.png}
\begin{eulercomment}
Perbedaan memungkinkan pemotongan satu objek dari objek lainnya.
Seperti persimpangan, ada bagian dari objek CSG Povray.
\end{eulercomment}
\begin{eulerprompt}
>povstart(light=[5,-5,5],fade=10);
\end{eulerprompt}
\begin{eulercomment}
Pada demonstrasi ini, kita mendefinisikan objek di Povray, alih-alih
menggunakan string di Euler. Definisinya akan tertulis di file secara
langsung.

Kotak koordinat dari -1 artinya [-1,-1,-1].
\end{eulercomment}
\begin{eulerprompt}
>povdefine("mycube",povbox(-1,1));
\end{eulerprompt}
\begin{eulercomment}
Kita dapat menggunkan objek ini di povobject(), yang akan
mengembalikan string seperti biasanya.
\end{eulercomment}
\begin{eulerprompt}
>c1=povobject("mycube",povlook(red));
\end{eulerprompt}
\begin{eulercomment}
Kita menghasilkan kubus kedua, dan merotasi dan menskalanya sedikit.
\end{eulercomment}
\begin{eulerprompt}
>c2=povobject("mycube",povlook(yellow),translate=[1,1,1], ...
>  rotate=xrotate(10°)+yrotate(10°), scale=1.2);
\end{eulerprompt}
\begin{eulercomment}
Lalu kita mengambil perbedaan dari kedua objek.
\end{eulercomment}
\begin{eulerprompt}
>writeln(povdifference(c1,c2));
\end{eulerprompt}
\begin{eulercomment}
Selanjutnya tambahkan tiga sumbu.
\end{eulercomment}
\begin{eulerprompt}
>writeAxis(-1.2,1.2,axis=1); ...
>writeAxis(-1.2,1.2,axis=2); ...
>writeAxis(-1.2,1.2,axis=4); ...
>povend();
\end{eulerprompt}
\eulerimg{27}{images/EMT4Plot3D_Icha Nur Oktaviani Hartono_23030630027-078.png}
\eulerheading{Fungsi Implisit}
\begin{eulercomment}
Povray dapat memplot himpunan dimana f(x,y,z)=0, sama seperti
parameter implisit pada plot3d. Hasilnya akan terlihat lebih baik.

Sintaks untuk fungsinya sedikit berbeda. Kalian tidak dapat
menggunakan output dari Maxima atau ekspresi Euler.

\end{eulercomment}
\begin{eulerformula}
\[
((x^2+y^2-c^2)^2+(z^2-1)^2)*((y^2+z^2-c^2)^2+(x^2-1)^2)*((z^2+x^2-c^2)^2+(y^2-1)^2)=d
\]
\end{eulerformula}
\begin{eulerprompt}
>povstart(angle=70°,height=50°,zoom=4);
>c=0.1; d=0.1; ...
>writeln(povsurface("(pow(pow(x,2)+pow(y,2)-pow(c,2),2)+pow(pow(z,2)-1,2))*(pow(pow(y,2)+pow(z,2)-pow(c,2),2)+pow(pow(x,2)-1,2))*(pow(pow(z,2)+pow(x,2)-pow(c,2),2)+pow(pow(y,2)-1,2))-d",povlook(red))); ...
>povend();
\end{eulerprompt}
\begin{euleroutput}
  Error : Povray error!
  
  Error generated by error() command
  
  povray:
      error("Povray error!");
  Try "trace errors" to inspect local variables after errors.
  povend:
      povray(file,w,h,aspect,exit); 
\end{euleroutput}
\begin{eulerprompt}
>povstart(angle=25°,height=10°); 
>writeln(povsurface("pow(x,2)+pow(y,2)*pow(z,2)-1",povlook(blue),povbox(-2,2,"")));
>povend();
\end{eulerprompt}
\eulerimg{27}{images/EMT4Plot3D_Icha Nur Oktaviani Hartono_23030630027-080.png}
\begin{eulerprompt}
>povstart(angle=70°,height=50°,zoom=4);
\end{eulerprompt}
\begin{eulercomment}
Membuat permukaan implisit. Ingat perbedaan sintaks di ekspresi.
\end{eulercomment}
\begin{eulerprompt}
>writeln(povsurface("pow(x,2)*y-pow(y,3)-pow(z,2)",povlook(green))); ...
>writeAxes(); ...
>povend();
\end{eulerprompt}
\eulerimg{27}{images/EMT4Plot3D_Icha Nur Oktaviani Hartono_23030630027-081.png}
\eulerheading{Objek Jaring}
\begin{eulercomment}
Pada contoh berikut, kita akab menampilkan bagaimana cara membuat
objek jaring, dan menggambarnya dengan informasi tambahan.

Kita ingin memaksimalkan xy, dibawah kondisi x+y=1 dan
mendemonstrasikan sentuhan tangensial dari garis level.
\end{eulercomment}
\begin{eulerprompt}
>povstart(angle=-10°,center=[0.5,0.5,0.5],zoom=7);
\end{eulerprompt}
\begin{eulercomment}
Kita tidak dapat menyimpan objek pada string seperti sebelumnya,
karena ukurannya terlalu besar. Jadi kita defnisiikan objek di file
Povray menggunakan #declare. Fungsi povtriangle() memporses hal ini
secara otomatis. Fungsi ini dapat menerima vektor normal sama seperti
pov3d().

Berikut definisi dari objek jaring dan menulisnya secara langsung di
file.
\end{eulercomment}
\begin{eulerprompt}
>x=0:0.02:1; y=x'; z=x*y; vx=-y; vy=-x; vz=1;
>mesh=povtriangles(x,y,z,"",vx,vy,vz);
\end{eulerprompt}
\begin{eulercomment}
Sekarang kita tentukan dua cakram, yang akan berpotongan dengan
permukaan.
\end{eulercomment}
\begin{eulerprompt}
>cl=povdisc([0.5,0.5,0],[1,1,0],2); ...
>ll=povdisc([0,0,1/4],[0,0,1],2);
\end{eulerprompt}
\begin{eulercomment}
Tuliskan permukaan dikurangi kedua cakram.
\end{eulercomment}
\begin{eulerprompt}
>writeln(povdifference(mesh,povunion([cl,ll]),povlook(green)));
\end{eulerprompt}
\begin{eulercomment}
Tulis kedua perpotongan tersebut.
\end{eulercomment}
\begin{eulerprompt}
>writeln(povintersection([mesh,cl],povlook(red))); ...
>writeln(povintersection([mesh,ll],povlook(gray)));
\end{eulerprompt}
\begin{eulercomment}
Tuliskan satu titik secara maksimal.
\end{eulercomment}
\begin{eulerprompt}
>writeln(povpoint([1/2,1/2,1/4],povlook(gray),size=2*defaultpointsize));
\end{eulerprompt}
\begin{eulercomment}
Tambahkan sumbu dan akhiri
\end{eulercomment}
\begin{eulerprompt}
>writeAxes(0,1,0,1,0,1,d=0.015); ...
>povend();
\end{eulerprompt}
\eulerimg{27}{images/EMT4Plot3D_Icha Nur Oktaviani Hartono_23030630027-082.png}
\eulerheading{Anaglyphs pada Povray}
\begin{eulercomment}
Untuk menghasilkan anaglyph bagi kacamata red/cyan, Povray harus
dijalankan sebanyak dua kali dari posisi kamera yang berbeda. Hal itu
menghasilkan dua file Povray dan dua file PNG, yang akan diproses
dengan fungsi loadanaglyph().

Tentu saja, kalian perlu kacamata rec/cyan untuk melihat contoh
berikut secara benar.

Fungsi pov3d() memiliki tombol sederhana untuk menghasilkan anaglyph.
\end{eulercomment}
\begin{eulerprompt}
>pov3d("-exp(-x^2-y^2)/2",r=2,height=45°,>anaglyph, ...
>  center=[0,0,0.5],zoom=3.5);
\end{eulerprompt}
\eulerimg{27}{images/EMT4Plot3D_Icha Nur Oktaviani Hartono_23030630027-083.png}
\begin{eulercomment}
Jika kalian membuat scene dengan objek, kalian harus menempatkan
pembuatan scene ke dalam suatu fungsi, dan menjalankannya dua kali
dengan nilai yang berbeda untuk parameter anaglyph.
\end{eulercomment}
\begin{eulerprompt}
>function myscene ...
\end{eulerprompt}
\begin{eulerudf}
    s=povsphere(povc,1);
    cl=povcylinder(-povz,povz,0.5);
    clx=povobject(cl,rotate=xrotate(90°));
    cly=povobject(cl,rotate=yrotate(90°));
    c=povbox([-1,-1,0],1);
    un=povunion([cl,clx,cly,c]);
    obj=povdifference(s,un,povlook(red));
    writeln(obj);
    writeAxes();
  endfunction
\end{eulerudf}
\begin{eulercomment}
Fungsi povanaglyph() melakukan semua ini. Parameternya seperti pada
povstart() dan povend() yang digabungkan.
\end{eulercomment}
\begin{eulerprompt}
>povanaglyph("myscene",zoom=4.5);
\end{eulerprompt}
\eulerimg{27}{images/EMT4Plot3D_Icha Nur Oktaviani Hartono_23030630027-084.png}
\eulerheading{Mendefinisikan Objek Sendiri}
\begin{eulercomment}
Antarmuka Povary Euler berisi banyak objek. Namun kalian tidak
dibatasi pada objek-objek tersebut. Kalian dapat membuat objek
sendiri, yang menggabungkan objek-objek lain, atau objek yang benar
benar baru.

Kami mendemonstrasikan sebuah torus. Perinth Povray untuk ini adalah
"torus". Jadi kita mengembalikan sebuah string dengan perintah ini dan
parameternya. Pehatikan bahwa torus selalu berpusat pada titik asal.
\end{eulercomment}
\begin{eulerprompt}
>function povdonat (r1,r2,look="") ...
\end{eulerprompt}
\begin{eulerudf}
    return "torus \{"+r1+","+r2+look+"\}";
  endfunction
\end{eulerudf}
\begin{eulercomment}
Berikut adalah torus pertama kita
\end{eulercomment}
\begin{eulerprompt}
>t1=povdonat(0.8,0.2)
\end{eulerprompt}
\begin{euleroutput}
  torus \{0.8,0.2\}
\end{euleroutput}
\begin{eulercomment}
Mari kita gunakan objek ini untuk membuat torus kedua, ditransasikan
dan diputar.
\end{eulercomment}
\begin{eulerprompt}
>t2=povobject(t1,rotate=xrotate(90°),translate=[0.8,0,0])
\end{eulerprompt}
\begin{euleroutput}
  object \{ torus \{0.8,0.2\}
   rotate 90 *x 
   translate <0.8,0,0>
   \}
\end{euleroutput}
\begin{eulercomment}
Now we place these objects into a scene. For the look, we use Phong
Shading.
\end{eulercomment}
\begin{eulerprompt}
>povstart(center=[0.4,0,0],angle=0°,zoom=3.8,aspect=1.5); ...
>writeln(povobject(t1,povlook(green,phong=1))); ...
>writeln(povobject(t2,povlook(green,phong=1))); ...
\end{eulerprompt}
\begin{eulerttcomment}
 >povend();
\end{eulerttcomment}
\begin{eulercomment}
memanggil program Povray. Namun, jika terjadi kesalahan, program ini
tidak menampilkan kesalahan. Oleh karen aitu, kalian harus menggunakan

\end{eulercomment}
\begin{eulerttcomment}
 >povend(<exit);
\end{eulerttcomment}
\begin{eulercomment}

jika ada yang tidak berhasil. Ini akan membiarkan jendela Povray tetap
terbuka.
\end{eulercomment}
\begin{eulerprompt}
>povend(h=320,w=480);
\end{eulerprompt}
\eulerimg{18}{images/EMT4Plot3D_Icha Nur Oktaviani Hartono_23030630027-085.png}
\begin{eulercomment}
Berikut adalah contoh yang lebih rumit. Kita menyelesaikan

\end{eulercomment}
\begin{eulerformula}
\[
Ax \le b, \quad x \ge 0, \quad c.x \to \text{Max.}
\]
\end{eulerformula}
\begin{eulercomment}
dan menunjukkan titik-titik yang layak dan optimal dalam plot 3D.
\end{eulercomment}
\begin{eulerprompt}
>A=[10,8,4;5,6,8;6,3,2;9,5,6];
>b=[10,10,10,10]';
>c=[1,1,1];
\end{eulerprompt}
\begin{eulercomment}
Pertama, mari kita cek jika contoh ini mempunyai solusi semua.
\end{eulercomment}
\begin{eulerprompt}
>x=simplex(A,b,c,>max,>check)'
\end{eulerprompt}
\begin{euleroutput}
  [0,  1,  0.5]
\end{euleroutput}
\begin{eulercomment}
Ternyata, contoh tersebut memiliki solusi.

Selanjutnya kita definisikan dua objek. Yang pertama merupakan bidang

\end{eulercomment}
\begin{eulerformula}
\[
a \cdot x \le b
\]
\end{eulerformula}
\begin{eulerprompt}
>function oneplane (a,b,look="") ...
\end{eulerprompt}
\begin{eulerudf}
    return povplane(a,b,look)
  endfunction
\end{eulerudf}
\begin{eulercomment}
Lalu kita definisikan perpotongan dari semua setengah ruang dan kubus.
\end{eulercomment}
\begin{eulerprompt}
>function adm (A, b, r, look="") ...
\end{eulerprompt}
\begin{eulerudf}
    ol=[];
    loop 1 to rows(A); ol=ol|oneplane(A[#],b[#]); end;
    ol=ol|povbox([0,0,0],[r,r,r]);
    return povintersection(ol,look);
  endfunction
\end{eulerudf}
\begin{eulercomment}
Sekarang kita dapat memplot scene
\end{eulercomment}
\begin{eulerprompt}
>povstart(angle=120°,center=[0.5,0.5,0.5],zoom=3.5); ...
>writeln(adm(A,b,2,povlook(green,0.4))); ...
>writeAxes(0,1.3,0,1.6,0,1.5); ...
\end{eulerprompt}
\begin{eulercomment}
Berikut adalah lingkaran di sekitar optimum.
\end{eulercomment}
\begin{eulerprompt}
>writeln(povintersection([povsphere(x,0.5),povplane(c,c.x')], ...
>  povlook(red,0.9)));
\end{eulerprompt}
\begin{eulercomment}
Dan kesalahan pada arah yang optimal.
\end{eulercomment}
\begin{eulerprompt}
>writeln(povarrow(x,c*0.5,povlook(red)));
\end{eulerprompt}
\begin{eulercomment}
Kita menambahkan teks ke layar. Teks hanyalah sebuah objek 3D. Kita
perlu menempatkan dan memutarnya sesuai dengan pandangan kita.
\end{eulercomment}
\begin{eulerprompt}
>writeln(povtext("Linear Problem",[0,0.2,1.3],size=0.05,rotate=5°)); ...
>povend();
\end{eulerprompt}
\eulerimg{27}{images/EMT4Plot3D_Icha Nur Oktaviani Hartono_23030630027-088.png}
\eulerheading{Lebih banyak contoh}
\begin{eulercomment}
Kalian dapat menemukan lebih banyak contoh untuk Povray di Euler pada
file berikut.

See: Examples/Dandelin Spheres\\
See: Examples/Donat Math\\
See: Examples/Trefoil Knot\\
See: Examples/Optimization by Affine Scaling

\begin{eulercomment}
\eulerheading{Soal Tambahan 1. Gambarlah bidang}
\begin{eulerformula}
\[
f(x,y)=3x+5y-10
\]
\end{eulerformula}
\begin{eulerprompt}
>plot3d("3*x+5*y-10",r=5):
\end{eulerprompt}
\eulerimg{27}{images/EMT4Plot3D_Icha Nur Oktaviani Hartono_23030630027-090.png}
\begin{eulercomment}
2. Gambarlah bidang\\
\end{eulercomment}
\begin{eulerformula}
\[
f(x,y)=x^4+y^5)
\]
\end{eulerformula}
\begin{eulercomment}
dengan distance = 3, zoom = 1, angle dan height = 0, dan aktifkan
fitur kontur
\end{eulercomment}
\begin{eulerprompt}
>plot3d("x^4+y^5",r=5,distance=3,zoom=1,angle=0,height=0,>contour):
\end{eulerprompt}
\eulerimg{27}{images/EMT4Plot3D_Icha Nur Oktaviani Hartono_23030630027-092.png}
\begin{eulerprompt}
>plot3d("x-(x^3/9)-y^2",height=45°,center=[0,0,0],>spectral):
\end{eulerprompt}
\eulerimg{27}{images/EMT4Plot3D_Icha Nur Oktaviani Hartono_23030630027-093.png}
\begin{eulerprompt}
>plot3d("x^2-y^2",>cp,cpcolor=blue,cpdelta=0.2):
\end{eulerprompt}
\eulerimg{27}{images/EMT4Plot3D_Icha Nur Oktaviani Hartono_23030630027-094.png}
\begin{eulerprompt}
>plot3d("sin(x^2*y)",>anaglyph,angle=30°):
\end{eulerprompt}
\eulerimg{27}{images/EMT4Plot3D_Icha Nur Oktaviani Hartono_23030630027-095.png}
\end{eulernotebook}
\end{document}
